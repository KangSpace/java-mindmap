\documentclass[a4paper,11pt]{article}
\usepackage{ulem}
\usepackage{amssymb,amsmath}
\usepackage{a4wide}
\usepackage[dvipsnames,svgnames]{xcolor}
\usepackage[pdftex]{graphicx}
\title{预览成卷}
\usepackage{hyperref}
% commands generated by html2latex


\begin{document}

2015-2016学年第一学期北京市广渠门中学高二数学期中试卷(理科)


\begin{center}

         年级:高中二年级\hspace*{0.333em}\hspace*{0.333em}\hspace*{0.333em}\hspace*{0.333em}
         学科:数学\hspace*{0.333em}\hspace*{0.333em}\hspace*{0.333em}\hspace*{0.333em}
         考试类型:期中考试\hspace*{0.333em}\hspace*{0.333em}\hspace*{0.333em}\hspace*{0.333em}
         年份:2015-2016\hspace*{0.333em}\hspace*{0.333em}\hspace*{0.333em}\hspace*{0.333em}
         地区:北京广渠门中学\\
\\

         总分:120分\hspace*{0.333em}\hspace*{0.333em}\hspace*{0.333em}\hspace*{0.333em}
         答题时间:120分钟\hspace*{0.333em}\hspace*{0.333em}\hspace*{0.333em}\hspace*{0.333em}
     
\end{center}

\subsubsection{}Get out, he said.
     

\textbf{一、选择题}

 \hspace*{0.333em}100023705\hspace*{0.333em}\hspace*{0.333em}\href{../question/viewQuestionDetail.jsp?questionID=100023705&amp;testPaperId=100229072}{查看}\hspace*{0.333em}\hspace*{0.333em}1.直线l\textsubscript{1}:4x+3y-1=0与直线l\textsubscript{2}:8x+6y+3=0的距离为(  )\\
\hspace*{0.333em}\hspace*{0.333em}(3.0分)\\
\hspace*{0.333em}\hspace*{0.333em}\hspace*{0.333em}\hspace*{0.333em}\hspace*{0.333em}\hspace*{0.333em}\hspace*{0.333em}\hspace*{0.333em}A:
\includegraphics{https://latex.codecogs.com/png.latex?\frac{2}{5}}‍\\
\hspace*{0.333em}\hspace*{0.333em}\hspace*{0.333em}\hspace*{0.333em}\hspace*{0.333em}\hspace*{0.333em}\hspace*{0.333em}\hspace*{0.333em}B:\frac\{1\}\{2\}\\
\hspace*{0.333em}\hspace*{0.333em}\hspace*{0.333em}\hspace*{0.333em}\hspace*{0.333em}\hspace*{0.333em}\hspace*{0.333em}\hspace*{0.333em}C:\frac\{1\}\{4\}‍\\
\hspace*{0.333em}\hspace*{0.333em}\hspace*{0.333em}\hspace*{0.333em}\hspace*{0.333em}\hspace*{0.333em}\hspace*{0.333em}\hspace*{0.333em}D: \$\frac\{1\}\{5\}\$ \\


 \hspace*{0.333em}100023710\hspace*{0.333em}\hspace*{0.333em}\href{../question/viewQuestionDetail.jsp?questionID=100023710&amp;testPaperId=100229072}{查看}\hspace*{0.333em}\hspace*{0.333em}2.在空间直角坐标系中,点B是点A(1,2,3)在坐标平面xOy上的射影,O为坐标原点,则OB的长为(  )\\
\hspace*{0.333em}\hspace*{0.333em}(3.0分)\\
\hspace*{0.333em}\hspace*{0.333em}\hspace*{0.333em}\hspace*{0.333em}\hspace*{0.333em}\hspace*{0.333em}\hspace*{0.333em}\hspace*{0.333em}A:\$\sqrt\{10\}\$\\
\hspace*{0.333em}\hspace*{0.333em}\hspace*{0.333em}\hspace*{0.333em}\hspace*{0.333em}\hspace*{0.333em}\hspace*{0.333em}\hspace*{0.333em}B:\sqrt\{13\}\\
\hspace*{0.333em}\hspace*{0.333em}\hspace*{0.333em}\hspace*{0.333em}\hspace*{0.333em}\hspace*{0.333em}\hspace*{0.333em}\hspace*{0.333em}C:\sqrt\{14\}‍\\
\hspace*{0.333em}\hspace*{0.333em}\hspace*{0.333em}\hspace*{0.333em}\hspace*{0.333em}\hspace*{0.333em}\hspace*{0.333em}\hspace*{0.333em}D:\sqrt\{5\}‍\\


 \hspace*{0.333em}100023712\hspace*{0.333em}\hspace*{0.333em}\href{../question/viewQuestionDetail.jsp?questionID=100023712&amp;testPaperId=100229072}{查看}\hspace*{0.333em}\hspace*{0.333em}3. 已知三棱锥A-BCD,E、F、G、H分别是AB、BC、CD、DA的中点,若AC=BD,则四边形EFGH为(  )\\
 \hspace*{0.333em}\hspace*{0.333em}(3.0分)\\
\hspace*{0.333em}\hspace*{0.333em}\hspace*{0.333em}\hspace*{0.333em}\hspace*{0.333em}\hspace*{0.333em}\hspace*{0.333em}\hspace*{0.333em}A: 梯形 \\
\\
\hspace*{0.333em}\hspace*{0.333em}\hspace*{0.333em}\hspace*{0.333em}\hspace*{0.333em}\hspace*{0.333em}\hspace*{0.333em}\hspace*{0.333em}B: 矩形\\
\\
\hspace*{0.333em}\hspace*{0.333em}\hspace*{0.333em}\hspace*{0.333em}\hspace*{0.333em}\hspace*{0.333em}\hspace*{0.333em}\hspace*{0.333em}C: 菱形 \\
\\
\hspace*{0.333em}\hspace*{0.333em}\hspace*{0.333em}\hspace*{0.333em}\hspace*{0.333em}\hspace*{0.333em}\hspace*{0.333em}\hspace*{0.333em}D: 正方形 \\
\\


 \hspace*{0.333em}100023716\hspace*{0.333em}\hspace*{0.333em}\href{../question/viewQuestionDetail.jsp?questionID=100023716&amp;testPaperId=100229072}{查看}\hspace*{0.333em}\hspace*{0.333em}4.在圆x\textsuperscript{2}+y\textsuperscript{2}-2x-6y=0内,过点E(0,1)的最长弦和最短弦分别为AC和BD,则四边形ABCD的面积为(  ) \\
\hspace*{0.333em}\hspace*{0.333em}(3.0分)\\
\hspace*{0.333em}\hspace*{0.333em}\hspace*{0.333em}\hspace*{0.333em}\hspace*{0.333em}\hspace*{0.333em}\hspace*{0.333em}\hspace*{0.333em}A:5\sqrt\{2\}‍\\
\hspace*{0.333em}\hspace*{0.333em}\hspace*{0.333em}\hspace*{0.333em}\hspace*{0.333em}\hspace*{0.333em}\hspace*{0.333em}\hspace*{0.333em}B:10\sqrt\{2\}‍\\
\hspace*{0.333em}\hspace*{0.333em}\hspace*{0.333em}\hspace*{0.333em}\hspace*{0.333em}\hspace*{0.333em}\hspace*{0.333em}\hspace*{0.333em}C:15\sqrt\{2\}‍\\
\hspace*{0.333em}\hspace*{0.333em}\hspace*{0.333em}\hspace*{0.333em}\hspace*{0.333em}\hspace*{0.333em}\hspace*{0.333em}\hspace*{0.333em}D:20\sqrt\{2\}\\


 \hspace*{0.333em}100023719\hspace*{0.333em}\hspace*{0.333em}\href{../question/viewQuestionDetail.jsp?questionID=100023719&amp;testPaperId=100229072}{查看}\hspace*{0.333em}\hspace*{0.333em}5.已知M=\{(x,y)|y=\sqrt\{9-x\textasciicircum2\},y\ne0\},N=\{(x,y)|y=x+b\},若M\capN\ne\varnothing,则b\in(  )\\
\hspace*{0.333em}\hspace*{0.333em}(3.0分)\\
\hspace*{0.333em}\hspace*{0.333em}\hspace*{0.333em}\hspace*{0.333em}\hspace*{0.333em}\hspace*{0.333em}\hspace*{0.333em}\hspace*{0.333em}A:[-3\sqrt\{2\},3\sqrt\{2\}]\\
\hspace*{0.333em}\hspace*{0.333em}\hspace*{0.333em}\hspace*{0.333em}\hspace*{0.333em}\hspace*{0.333em}\hspace*{0.333em}\hspace*{0.333em}B:(-3\sqrt\{2\},3\sqrt\{2\})\\
\\
\hspace*{0.333em}\hspace*{0.333em}\hspace*{0.333em}\hspace*{0.333em}\hspace*{0.333em}\hspace*{0.333em}\hspace*{0.333em}\hspace*{0.333em}C:(-3,3\sqrt\{2\}]\\
\hspace*{0.333em}\hspace*{0.333em}\hspace*{0.333em}\hspace*{0.333em}\hspace*{0.333em}\hspace*{0.333em}\hspace*{0.333em}\hspace*{0.333em}D:[-3,3\sqrt\{2\}]\\


 \hspace*{0.333em}100023725\hspace*{0.333em}\hspace*{0.333em}\href{../question/viewQuestionDetail.jsp?questionID=100023725&amp;testPaperId=100229072}{查看}\hspace*{0.333em}\hspace*{0.333em}6.圆心为(2,-1)的圆,在直线x-y-1=0上截得的弦长为2\sqrt\{2\},那么,这个圆的方程为(  ) \\
\hspace*{0.333em}\hspace*{0.333em}(3.0分)\\
\hspace*{0.333em}\hspace*{0.333em}\hspace*{0.333em}\hspace*{0.333em}\hspace*{0.333em}\hspace*{0.333em}\hspace*{0.333em}\hspace*{0.333em}A: (x-2)\textsuperscript{2}+(y+1)\textsuperscript{2}=4 \\
\\
\hspace*{0.333em}\hspace*{0.333em}\hspace*{0.333em}\hspace*{0.333em}\hspace*{0.333em}\hspace*{0.333em}\hspace*{0.333em}\hspace*{0.333em}B: (x-2)\textsuperscript{2}+(y+1)\textsuperscript{2}=2\\
\\
\hspace*{0.333em}\hspace*{0.333em}\hspace*{0.333em}\hspace*{0.333em}\hspace*{0.333em}\hspace*{0.333em}\hspace*{0.333em}\hspace*{0.333em}C: (x+2)\textsuperscript{2}+(y-1)\textsuperscript{2}=4\\
\\
\hspace*{0.333em}\hspace*{0.333em}\hspace*{0.333em}\hspace*{0.333em}\hspace*{0.333em}\hspace*{0.333em}\hspace*{0.333em}\hspace*{0.333em}D: (x+2)\textsuperscript{2}+(y-1)\textsuperscript{2}=2 \\
\\


 \hspace*{0.333em}100023727\hspace*{0.333em}\hspace*{0.333em}\href{../question/viewQuestionDetail.jsp?questionID=100023727&amp;testPaperId=100229072}{查看}\hspace*{0.333em}\hspace*{0.333em}7.在三棱锥P-ABC中,PA\bot底面ABC,BC\botAC,\angleABC=30\textasciicircum\{\circ\},AC=1,PB=2\sqrt\{3\},则PC与平面PAB所成余弦值是(  ) \\
\hspace*{0.333em}\hspace*{0.333em}(3.0分)\\
\hspace*{0.333em}\hspace*{0.333em}\hspace*{0.333em}\hspace*{0.333em}\hspace*{0.333em}\hspace*{0.333em}\hspace*{0.333em}\hspace*{0.333em}A:\frac\{\sqrt\{33\}\}\{6\}\\
\hspace*{0.333em}\hspace*{0.333em}\hspace*{0.333em}\hspace*{0.333em}\hspace*{0.333em}\hspace*{0.333em}\hspace*{0.333em}\hspace*{0.333em}B:\frac\{\sqrt\{3\}\}\{3\}\\
\hspace*{0.333em}\hspace*{0.333em}\hspace*{0.333em}\hspace*{0.333em}\hspace*{0.333em}\hspace*{0.333em}\hspace*{0.333em}\hspace*{0.333em}C:\frac\{\sqrt\{3\}\}\{6\}\\
\hspace*{0.333em}\hspace*{0.333em}\hspace*{0.333em}\hspace*{0.333em}\hspace*{0.333em}\hspace*{0.333em}\hspace*{0.333em}\hspace*{0.333em}D:\frac\{\sqrt\{6\}\}\{3\}\\


 \hspace*{0.333em}100023728\hspace*{0.333em}\hspace*{0.333em}\href{../question/viewQuestionDetail.jsp?questionID=100023728&amp;testPaperId=100229072}{查看}\hspace*{0.333em}\hspace*{0.333em}8.经过点P(0,-1)作直线l,若直线l与连接A(1,-2),B(2,1)的线段没有公共点,则直线l的斜率k与倾斜角\alpha的取值范围分别是(  ) \\
\hspace*{0.333em}\hspace*{0.333em}(3.0分)\\
\hspace*{0.333em}\hspace*{0.333em}\hspace*{0.333em}\hspace*{0.333em}\hspace*{0.333em}\hspace*{0.333em}\hspace*{0.333em}\hspace*{0.333em}A:(-\infty,-1)\cup(1,+\infty),(\frac\{\pi\}\{4\},\frac\{3\pi\}\{4\}) \\
\\
\hspace*{0.333em}\hspace*{0.333em}\hspace*{0.333em}\hspace*{0.333em}\hspace*{0.333em}\hspace*{0.333em}\hspace*{0.333em}\hspace*{0.333em}B:(-\infty,-1)\cup(1,+\infty),(\frac\{\pi\}\{4\},\frac\{\pi\}\{2\})\cup(\frac\{\pi\}\{2\},\frac\{3\pi\}\{4\}) \\
\\
\hspace*{0.333em}\hspace*{0.333em}\hspace*{0.333em}\hspace*{0.333em}\hspace*{0.333em}\hspace*{0.333em}\hspace*{0.333em}\hspace*{0.333em}C:(-1,1),[\frac\{\pi\}\{4\} ,\frac\{3\pi\}\{4\}]\\
\\
\hspace*{0.333em}\hspace*{0.333em}\hspace*{0.333em}\hspace*{0.333em}\hspace*{0.333em}\hspace*{0.333em}\hspace*{0.333em}\hspace*{0.333em}D:(-1,1),[0,\frac\{\pi\}\{4\}]\cup[\frac\{3\pi\}\{4\},0) \\
\\


\textbf{二、填空题}

 \hspace*{0.333em}100023733\hspace*{0.333em}\hspace*{0.333em}\href{../question/viewQuestionDetail.jsp?questionID=100023733&amp;testPaperId=100229072}{查看}\hspace*{0.333em}\hspace*{0.333em}9.过点(2,3)且在两坐标轴上的截距的绝对值相等的直线方程为. \\
\hspace*{0.333em}\hspace*{0.333em}(3.0分)\\


 \hspace*{0.333em}100023734\hspace*{0.333em}\hspace*{0.333em}\href{../question/viewQuestionDetail.jsp?questionID=100023734&amp;testPaperId=100229072}{查看}\hspace*{0.333em}\hspace*{0.333em}10.已知点M(a,b)在直线3x+4y=15上,则a\textsuperscript{2}+b\textsuperscript{2}的最小值为.\\
\hspace*{0.333em}\hspace*{0.333em}(3.0分)\\


 \hspace*{0.333em}100023735\hspace*{0.333em}\hspace*{0.333em}\href{../question/viewQuestionDetail.jsp?questionID=100023735&amp;testPaperId=100229072}{查看}\hspace*{0.333em}\hspace*{0.333em}11.当0$<$k$<$\frac\{1\}\{2\}时,两条直线kx-y=k-1、ky-x=2k的交点在象限.\\
\hspace*{0.333em}\hspace*{0.333em}(3.0分)\\


 \hspace*{0.333em}100023752\hspace*{0.333em}\hspace*{0.333em}\href{../question/viewQuestionDetail.jsp?questionID=100023752&amp;testPaperId=100229072}{查看}\hspace*{0.333em}\hspace*{0.333em}12.过圆:x\textsuperscript{2}+y\textsuperscript{2}=r\textsuperscript{2}外一点P(x\textsubscript{0},y\textsubscript{0})引此圆的两条切线,切点为A、B,则直线AB的方程为\_\_\_\_\_\_\_\_\_\_\_\_\_\_\_\_\_.\\
\hspace*{0.333em}\hspace*{0.333em}(3.0分)\\


 \hspace*{0.333em}100023760\hspace*{0.333em}\hspace*{0.333em}\href{../question/viewQuestionDetail.jsp?questionID=100023760&amp;testPaperId=100229072}{查看}\hspace*{0.333em}\hspace*{0.333em}13.圆x\textsuperscript{2}+y\textsuperscript{2}+2x+4y-3=0上到直线x+y+1=0的距离为 的共有个. \\
\hspace*{0.333em}\hspace*{0.333em}(3.0分)\\


 \hspace*{0.333em}

 \hspace*{0.333em}

 \hspace*{0.333em}

 \hspace*{0.333em}

 \hspace*{0.333em}

\end{document}
