\documentclass[]{article}
\usepackage{lmodern}
\usepackage{amssymb,amsmath}
\usepackage{ifxetex,ifluatex}
\usepackage{fixltx2e} % provides \textsubscript
\ifnum 0\ifxetex 1\fi\ifluatex 1\fi=0 % if pdftex
  \usepackage[T1]{fontenc}
  \usepackage[utf8]{inputenc}
\else % if luatex or xelatex
  \ifxetex
    \usepackage{mathspec}
  \else
    \usepackage{fontspec}
  \fi
  \defaultfontfeatures{Ligatures=TeX,Scale=MatchLowercase}
\fi
% use upquote if available, for straight quotes in verbatim environments
\IfFileExists{upquote.sty}{\usepackage{upquote}}{}
% use microtype if available
\IfFileExists{microtype.sty}{%
\usepackage[]{microtype}
\UseMicrotypeSet[protrusion]{basicmath} % disable protrusion for tt fonts
}{}
\PassOptionsToPackage{hyphens}{url} % url is loaded by hyperref
\usepackage[unicode=true]{hyperref}
\hypersetup{
            pdftitle={预览成卷},
            pdfborder={0 0 0},
            breaklinks=true}
\urlstyle{same}  % don't use monospace font for urls
\IfFileExists{parskip.sty}{%
\usepackage{parskip}
}{% else
\setlength{\parindent}{0pt}
\setlength{\parskip}{6pt plus 2pt minus 1pt}
}
\setlength{\emergencystretch}{3em}  % prevent overfull lines
\providecommand{\tightlist}{%
  \setlength{\itemsep}{0pt}\setlength{\parskip}{0pt}}
\setcounter{secnumdepth}{0}
% Redefines (sub)paragraphs to behave more like sections
\ifx\paragraph\undefined\else
\let\oldparagraph\paragraph
\renewcommand{\paragraph}[1]{\oldparagraph{#1}\mbox{}}
\fi
\ifx\subparagraph\undefined\else
\let\oldsubparagraph\subparagraph
\renewcommand{\subparagraph}[1]{\oldsubparagraph{#1}\mbox{}}
\fi

% set default figure placement to htbp
\makeatletter
\def\fps@figure{htbp}
\makeatother


\title{预览成卷}
\date{}

\begin{document}
\maketitle

2015-2016学年第一学期北京市广渠门中学高二数学期中试卷(理科)

年级:高中二年级~~~~ 学科:数学~~~~ 考试类型:期中考试~~~~
年份:2015-2016~~~~ 地区:北京广渠门中学 总分:120分~~~~
答题时间:120分钟~~~~

\subsubsection{}\label{section}

\hypertarget{questionArea}{}
\textbf{一、选择题}

~{{100023705}~~\href{../question/viewQuestionDetail.jsp?questionID=100023705\&testPaperId=100229072}{查看}~~1.直线l\textsubscript{1}:4x+3y-1=0与直线l\textsubscript{2}:8x+6y+3=0的距离为(  )\\
\hspace*{0.333em}\hspace*{0.333em}(3.0分)\\
\hspace*{0.333em}\hspace*{0.333em}\hspace*{0.333em}\hspace*{0.333em}\hspace*{0.333em}\hspace*{0.333em}\hspace*{0.333em}\hspace*{0.333em}A:$\frac{2}{5}$\protect\hypertarget{_baidu_bookmark_start_0}{}{‍}\\
\hspace*{0.333em}\hspace*{0.333em}\hspace*{0.333em}\hspace*{0.333em}\hspace*{0.333em}\hspace*{0.333em}\hspace*{0.333em}\hspace*{0.333em}B:$\frac{1}{2}$\\
\hspace*{0.333em}\hspace*{0.333em}\hspace*{0.333em}\hspace*{0.333em}\hspace*{0.333em}\hspace*{0.333em}\hspace*{0.333em}\hspace*{0.333em}C:$\frac{1}{4}$\protect\hypertarget{_baidu_bookmark_start_2}{}{‍}\\
\hspace*{0.333em}\hspace*{0.333em}\hspace*{0.333em}\hspace*{0.333em}\hspace*{0.333em}\hspace*{0.333em}\hspace*{0.333em}\hspace*{0.333em}D:$\frac{1}{5}$\\
}

~{{100023710}~~\href{../question/viewQuestionDetail.jsp?questionID=100023710\&testPaperId=100229072}{查看}~~2.在空间直角坐标系中,点B是点A(1,2,3)在坐标平面xOy上的射影,O为坐标原点,则OB的长为(  )\\
\hspace*{0.333em}\hspace*{0.333em}(3.0分)\\
\hspace*{0.333em}\hspace*{0.333em}\hspace*{0.333em}\hspace*{0.333em}\hspace*{0.333em}\hspace*{0.333em}\hspace*{0.333em}\hspace*{0.333em}A:{\$\textbackslash{}sqrt\{10\}\$}\\
\hspace*{0.333em}\hspace*{0.333em}\hspace*{0.333em}\hspace*{0.333em}\hspace*{0.333em}\hspace*{0.333em}\hspace*{0.333em}\hspace*{0.333em}B:{\textbackslash{}sqrt\{13\}}\\
\hspace*{0.333em}\hspace*{0.333em}\hspace*{0.333em}\hspace*{0.333em}\hspace*{0.333em}\hspace*{0.333em}\hspace*{0.333em}\hspace*{0.333em}C:{\textbackslash{}sqrt\{14\}}\protect\hypertarget{_baidu_bookmark_start_2}{}{‍}\\
\hspace*{0.333em}\hspace*{0.333em}\hspace*{0.333em}\hspace*{0.333em}\hspace*{0.333em}\hspace*{0.333em}\hspace*{0.333em}\hspace*{0.333em}D:{\textbackslash{}sqrt\{5\}}\protect\hypertarget{_baidu_bookmark_start_3}{}{‍}\\
}

~{{100023712}~~\href{../question/viewQuestionDetail.jsp?questionID=100023712\&testPaperId=100229072}{查看}~~3.
已知三棱锥A-BCD,E、F、G、H分别是AB、BC、CD、DA的中点,若AC=BD,则四边形EFGH为(  )\\
\hspace*{0.333em}\hspace*{0.333em}(3.0分)\\
\hspace*{0.333em}\hspace*{0.333em}\hspace*{0.333em}\hspace*{0.333em}\hspace*{0.333em}\hspace*{0.333em}\hspace*{0.333em}\hspace*{0.333em}A:
梯形\\[2\baselineskip]\hspace*{0.333em}\hspace*{0.333em}\hspace*{0.333em}\hspace*{0.333em}\hspace*{0.333em}\hspace*{0.333em}\hspace*{0.333em}\hspace*{0.333em}B:
矩形\\[2\baselineskip]\hspace*{0.333em}\hspace*{0.333em}\hspace*{0.333em}\hspace*{0.333em}\hspace*{0.333em}\hspace*{0.333em}\hspace*{0.333em}\hspace*{0.333em}C:
菱形\\[2\baselineskip]\hspace*{0.333em}\hspace*{0.333em}\hspace*{0.333em}\hspace*{0.333em}\hspace*{0.333em}\hspace*{0.333em}\hspace*{0.333em}\hspace*{0.333em}D:
正方形\\[2\baselineskip]}

~{{100023716}~~\href{../question/viewQuestionDetail.jsp?questionID=100023716\&testPaperId=100229072}{查看}~~4.在圆x\textsuperscript{2}+y\textsuperscript{2}-2x-6y=0内,过点E(0,1)的最长弦和最短弦分别为AC和BD,则四边形ABCD的面积为(  )\\
\hspace*{0.333em}\hspace*{0.333em}(3.0分)\\
\hspace*{0.333em}\hspace*{0.333em}\hspace*{0.333em}\hspace*{0.333em}\hspace*{0.333em}\hspace*{0.333em}\hspace*{0.333em}\hspace*{0.333em}A:5{\textbackslash{}sqrt\{2\}}\protect\hypertarget{_baidu_bookmark_start_0}{}{‍}\\
\hspace*{0.333em}\hspace*{0.333em}\hspace*{0.333em}\hspace*{0.333em}\hspace*{0.333em}\hspace*{0.333em}\hspace*{0.333em}\hspace*{0.333em}B:10{\textbackslash{}sqrt\{2\}}\protect\hypertarget{_baidu_bookmark_start_1}{}{‍}\\
\hspace*{0.333em}\hspace*{0.333em}\hspace*{0.333em}\hspace*{0.333em}\hspace*{0.333em}\hspace*{0.333em}\hspace*{0.333em}\hspace*{0.333em}C:15{\textbackslash{}sqrt\{2\}}\protect\hypertarget{_baidu_bookmark_start_2}{}{‍}\\
\hspace*{0.333em}\hspace*{0.333em}\hspace*{0.333em}\hspace*{0.333em}\hspace*{0.333em}\hspace*{0.333em}\hspace*{0.333em}\hspace*{0.333em}D:20{\textbackslash{}sqrt\{2\}}\\
}

~{{100023719}~~\href{../question/viewQuestionDetail.jsp?questionID=100023719\&testPaperId=100229072}{查看}~~5.已知M=\{(x,y)\textbar{}y={\textbackslash{}sqrt\{9-x\^{}2\}},y{\textbackslash{}ne}0{\}},N=\{(x,y)\textbar{}y=x+b\},若M{\textbackslash{}cap}N{\textbackslash{}ne}{\textbackslash{}varnothing},则b{\textbackslash{}in}(  )\\
\hspace*{0.333em}\hspace*{0.333em}(3.0分)\\
\hspace*{0.333em}\hspace*{0.333em}\hspace*{0.333em}\hspace*{0.333em}\hspace*{0.333em}\hspace*{0.333em}\hspace*{0.333em}\hspace*{0.333em}A:[-3{\textbackslash{}sqrt\{2\}},3{\textbackslash{}sqrt\{2\}}]\\
\hspace*{0.333em}\hspace*{0.333em}\hspace*{0.333em}\hspace*{0.333em}\hspace*{0.333em}\hspace*{0.333em}\hspace*{0.333em}\hspace*{0.333em}B:(-3{\textbackslash{}sqrt\{2\}},3{\textbackslash{}sqrt\{2\}})\\[2\baselineskip]\hspace*{0.333em}\hspace*{0.333em}\hspace*{0.333em}\hspace*{0.333em}\hspace*{0.333em}\hspace*{0.333em}\hspace*{0.333em}\hspace*{0.333em}C:(-3,3{\textbackslash{}sqrt\{2\}}]\\
\hspace*{0.333em}\hspace*{0.333em}\hspace*{0.333em}\hspace*{0.333em}\hspace*{0.333em}\hspace*{0.333em}\hspace*{0.333em}\hspace*{0.333em}D:[-3,3{\textbackslash{}sqrt\{2\}}]\\
}

~{{100023725}~~\href{../question/viewQuestionDetail.jsp?questionID=100023725\&testPaperId=100229072}{查看}~~6.圆心为(2,-1)的圆,在直线x-y-1=0上截得的弦长为2{\textbackslash{}sqrt\{2\}},那么,这个圆的方程为(  )\\
\hspace*{0.333em}\hspace*{0.333em}(3.0分)\\
\hspace*{0.333em}\hspace*{0.333em}\hspace*{0.333em}\hspace*{0.333em}\hspace*{0.333em}\hspace*{0.333em}\hspace*{0.333em}\hspace*{0.333em}A:
(x-2)\textsuperscript{2}+(y+1)\textsuperscript{2}=4\\[2\baselineskip]\hspace*{0.333em}\hspace*{0.333em}\hspace*{0.333em}\hspace*{0.333em}\hspace*{0.333em}\hspace*{0.333em}\hspace*{0.333em}\hspace*{0.333em}B:
(x-2)\textsuperscript{2}+(y+1)\textsuperscript{2}=2\\[2\baselineskip]\hspace*{0.333em}\hspace*{0.333em}\hspace*{0.333em}\hspace*{0.333em}\hspace*{0.333em}\hspace*{0.333em}\hspace*{0.333em}\hspace*{0.333em}C:
(x+2)\textsuperscript{2}+(y-1)\textsuperscript{2}=4\\[2\baselineskip]\hspace*{0.333em}\hspace*{0.333em}\hspace*{0.333em}\hspace*{0.333em}\hspace*{0.333em}\hspace*{0.333em}\hspace*{0.333em}\hspace*{0.333em}D:
(x+2)\textsuperscript{2}+(y-1)\textsuperscript{2}=2\\[2\baselineskip]}

~{{100023727}~~\href{../question/viewQuestionDetail.jsp?questionID=100023727\&testPaperId=100229072}{查看}~~7.在三棱锥P-ABC中,PA{\textbackslash{}bot}底面ABC,BC{\textbackslash{}bot}AC,{\textbackslash{}angle}ABC=30{\^{}\{\textbackslash{}circ\}},AC=1,PB=2{\textbackslash{}sqrt\{3\}},则PC与平面PAB所成余弦值是(  )\\
\hspace*{0.333em}\hspace*{0.333em}(3.0分)\\
\hspace*{0.333em}\hspace*{0.333em}\hspace*{0.333em}\hspace*{0.333em}\hspace*{0.333em}\hspace*{0.333em}\hspace*{0.333em}\hspace*{0.333em}A:{\textbackslash{}frac\{\textbackslash{}sqrt\{33\}\}\{6\}}\\
\hspace*{0.333em}\hspace*{0.333em}\hspace*{0.333em}\hspace*{0.333em}\hspace*{0.333em}\hspace*{0.333em}\hspace*{0.333em}\hspace*{0.333em}B:{\textbackslash{}frac\{\textbackslash{}sqrt\{3\}\}\{3\}}\\
\hspace*{0.333em}\hspace*{0.333em}\hspace*{0.333em}\hspace*{0.333em}\hspace*{0.333em}\hspace*{0.333em}\hspace*{0.333em}\hspace*{0.333em}C:{\textbackslash{}frac\{\textbackslash{}sqrt\{3\}\}\{6\}}\\
\hspace*{0.333em}\hspace*{0.333em}\hspace*{0.333em}\hspace*{0.333em}\hspace*{0.333em}\hspace*{0.333em}\hspace*{0.333em}\hspace*{0.333em}D:{\textbackslash{}frac\{\textbackslash{}sqrt\{6\}\}\{3\}}\\
}

~{{100023728}~~\href{../question/viewQuestionDetail.jsp?questionID=100023728\&testPaperId=100229072}{查看}~~8.经过点P(0,-1)作直线l,若直线l与连接A(1,-2),B(2,1)的线段没有公共点,则直线l的斜率k与倾斜角{\textbackslash{}alpha}的取值范围分别是(  )\\
\hspace*{0.333em}\hspace*{0.333em}(3.0分)\\
\hspace*{0.333em}\hspace*{0.333em}\hspace*{0.333em}\hspace*{0.333em}\hspace*{0.333em}\hspace*{0.333em}\hspace*{0.333em}\hspace*{0.333em}A:(-{\textbackslash{}infty},-1){\textbackslash{}cup}(1,+{\textbackslash{}infty}),({\textbackslash{}frac\{\textbackslash{}pi\}\{4\}},{\textbackslash{}frac\{3\textbackslash{}pi\}\{4\}})\\[2\baselineskip]\hspace*{0.333em}\hspace*{0.333em}\hspace*{0.333em}\hspace*{0.333em}\hspace*{0.333em}\hspace*{0.333em}\hspace*{0.333em}\hspace*{0.333em}B:(-{\textbackslash{}infty},-1){\textbackslash{}cup}(1,+{\textbackslash{}infty}),({\textbackslash{}frac\{\textbackslash{}pi\}\{4\}},{\textbackslash{}frac\{\textbackslash{}pi\}\{2\}}){\textbackslash{}cup}({\textbackslash{}frac\{\textbackslash{}pi\}\{2\}},{\textbackslash{}frac\{3\textbackslash{}pi\}\{4\}})\\[2\baselineskip]\hspace*{0.333em}\hspace*{0.333em}\hspace*{0.333em}\hspace*{0.333em}\hspace*{0.333em}\hspace*{0.333em}\hspace*{0.333em}\hspace*{0.333em}C:(-1,1),{[}{\textbackslash{}frac\{\textbackslash{}pi\}\{4\}}
,{\textbackslash{}frac\{3\textbackslash{}pi\}\{4\}}{]}\\[2\baselineskip]\hspace*{0.333em}\hspace*{0.333em}\hspace*{0.333em}\hspace*{0.333em}\hspace*{0.333em}\hspace*{0.333em}\hspace*{0.333em}\hspace*{0.333em}D:(-1,1),{[}0,{\textbackslash{}frac\{\textbackslash{}pi\}\{4\}}{]}{\textbackslash{}cup}{[}{\textbackslash{}frac\{3\textbackslash{}pi\}\{4\}},0)\\[2\baselineskip]}

\textbf{二、填空题}

~{{100023733}~~\href{../question/viewQuestionDetail.jsp?questionID=100023733\&testPaperId=100229072}{查看}~~9.过点(2,3)且在两坐标轴上的截距的绝对值相等的直线方程为.\\
\hspace*{0.333em}\hspace*{0.333em}(3.0分)\\
}

~{{100023734}~~\href{../question/viewQuestionDetail.jsp?questionID=100023734\&testPaperId=100229072}{查看}~~10.已知点M(a,b)在直线3x+4y=15上,则a\textsuperscript{2}+b\textsuperscript{2}的最小值为.\\
\hspace*{0.333em}\hspace*{0.333em}(3.0分)\\
}

~{{100023735}~~\href{../question/viewQuestionDetail.jsp?questionID=100023735\&testPaperId=100229072}{查看}~~11.当0\textless{}k\textless{}{\textbackslash{}frac\{1\}\{2\}}时,两条直线kx-y=k-1、ky-x=2k的交点在象限.\\
\hspace*{0.333em}\hspace*{0.333em}(3.0分)\\
}

~{{100023752}~~\href{../question/viewQuestionDetail.jsp?questionID=100023752\&testPaperId=100229072}{查看}~~12.过圆:x\textsuperscript{2}+y\textsuperscript{2}=r\textsuperscript{2}外一点P(x\textsubscript{0},y\textsubscript{0})引此圆的两条切线,切点为A、B,则直线AB的方程为\_\_\_\_\_\_\_\_\_\_\_\_\_\_\_\_\_.\\
\hspace*{0.333em}\hspace*{0.333em}(3.0分)\\
}

~{{100023760}~~\href{../question/viewQuestionDetail.jsp?questionID=100023760\&testPaperId=100229072}{查看}~~13.圆x\textsuperscript{2}+y\textsuperscript{2}+2x+4y-3=0上到直线x+y+1=0的距离为
的共有个.\\
\hspace*{0.333em}\hspace*{0.333em}(3.0分)\\
}

~

~

~

~

~

\end{document}
